%
\section{Discussion and Additional Results}\label{app:add_results}
%

\paragraph{Vocabulary size scaling}
%

Table \ref{scaling_vsize} showcases interesting correlation between associative recall performance for varying vocabulary sizes and loss on the {\sc The Pile}. In this case, we fix sequence length for associative recall to be $2048$, the same sequence length used to train all models on the {\sc The Pile}.

We observe a similar phenomenon on other slices of tasks from our mechanistic design benchmarks, indicating that it may be possible to derive predictive laws for performance at scale, based on fast experimentation on synthetic tasks with models of $1$ or $2$ layers. Surprisingly, performance on our language synthetics appears to be further linked to performance as attention replacement in other domains (Appendix \ref{appendix:image-classification} for results on image classification).

\begin{table}[!bh]
\small
\centering
\caption{{\sf Hyena} Accuracy on associative recall with varying vocabulary size $10$, $20$, $30$, $40$ in relation to test loss on {\sc The Pile} after $5$ billion tokens. We notice a correlation between the two performance metrics, suggesting that slices of our mechanistic design synthetics may be potentially predictive of performance at scale.}
\vspace{2mm}
\label{scaling_vsize}
\setlength{\tabcolsep}{4pt}
\begin{tabular}{@{}c|ccccc@{}}
\toprule
Model & Acc @ $10$ & Acc @ $20$ & Acc @ $30$ & Acc @ $40$ & Loss @ $5$B on {\sc The Pile} \\
\midrule 
Conv1d & $32$ & $11$ & $10$ & $8$ & $4.21$\\ 
AFT-conv & $55$ & $21$ & $12$ & $10$ & $3.57$\\ 
H3 & $92$ & $60$ & $13$ & $10$ & $2.69$\\
Transformer & $100$ & $100$ & $92$ & $82$ & $2.59$\\ 
{\sf Hyena} & $100$ & $100$ & $98$ & $85$ & $2.59$\\ 
\bottomrule
\end{tabular}
\end{table}

\paragraph{Single layer recall}
%

All experiments on our synthetic tasks default to $2$ layer models. We choose $2$ as it is the canonical number for mechanistic analysis of Transformers \citep{elhage2021mathematical} based on \textit{circuits}. Interestingly, a single layer of {\sf Hyena} (width $64$) is capable of performing associative recall, solving the task completely even in the challenging setting with vocabulary size $40$. Reverse engineering exactly how the single {\sf Hyena} operator is able to perform recall is left for future work.

\subsection{Learning Arithmetic}
%
We showcase an additional task in our mechanistic design benchmark: learning arithmetic. We train {\sf Hyena} models of increasing depth ($1$, $2$ and $3$ layers) on a dataset of $D_n$-digit addition. As an example, a $3$-digit addition input sample is given by the sequence
\[ 
    {\tt 1, 2, 3, 9, 5, 4, 1, 0, 7, 7}
\]
where the first $6$ digits contain the two $3$ digits numbers to add, and the last $4$ the result. Our models are optimized using standard autoregressive training i.e., predicting the next token, since they are causal. In particular, we optimize models to learn a map $x \mapsto y$ where $x$ is the original prompt without the last element, and $y$ equal to $x$ shifted right by one position. We mask the first $2 D_n - 1$ elements of the loss for each sequence since they contain predictions for addends and not results.

We report results in Figure \ref{fig:arithmetic}. A single layer of {\sf Hyena} is able to learn to perform addition with up to $4$ digits. Longer numbers require deeper models. In our experiments, alternative architectures such as AFT-conv struggle to learn arithmetic, signaling a cap in capability.

%
\begin{figure}
    \centering
    % This file was created with tikzplotlib v0.10.1.
\begin{tikzpicture}

\definecolor{darkgray176}{RGB}{176,176,176}
\definecolor{darkorange25512714}{RGB}{255,127,14}
\definecolor{steelblue31119180}{RGB}{31,119,180}
\definecolor{lightseagreen}{RGB}{32,178,170}

\begin{groupplot}[group style={group size=4 by 3, vertical sep=1.5cm}]
\nextgroupplot[
width=.28\linewidth,
xmin=0, xmax=80, ymin=0, ymax=3,
xlabel = {\sf \textit{epochs}}, xlabel style={at={(.5,-.12)}},
title={\sf \textbf{Layers}: 1, \textbf{Digits}: 2},
]
\addplot [line width=1.5pt, steelblue31119180]
table {%
0 19.8500556945801
1 14.6964349746704
2 7.82520723342896
3 3.55504679679871
4 2.27022767066956
5 1.65402841567993
6 1.01785886287689
7 0.558491230010986
8 0.368733167648315
9 0.291138887405396
10 0.211906835436821
11 0.187336325645447
12 0.170502096414566
13 0.157181560993195
14 0.158629402518272
15 0.142874717712402
16 0.142208084464073
17 0.14371845126152
18 0.145146876573563
19 0.151019930839539
20 0.142341703176498
21 0.1413953602314
22 0.142300367355347
23 0.13967701792717
24 0.139571860432625
25 0.145397439599037
26 0.142405033111572
27 0.151020094752312
28 0.1421257853508
29 0.14111365377903
30 0.143797650933266
31 0.14401076734066
32 0.142815381288528
33 0.142388001084328
34 0.142405718564987
35 0.144764810800552
36 0.143118128180504
37 0.142737194895744
38 0.14590784907341
39 0.142867088317871
40 0.144714966416359
41 0.141950070858002
42 0.144169852137566
43 0.142712444067001
44 0.142708286643028
45 0.144108429551125
46 0.143156096339226
47 0.143393129110336
48 0.144469887018204
49 0.143071204423904
50 0.144044503569603
51 0.144044890999794
52 0.143974304199219
53 0.144277423620224
54 0.143858775496483
55 0.144573271274567
56 0.144436970353127
57 0.143984705209732
58 0.144162073731422
59 0.144311249256134
60 0.144347608089447
61 0.144232347607613
62 0.14429560303688
63 0.144255951046944
64 0.144344061613083
65 0.144371554255486
66 0.144388929009438
67 0.144321799278259
68 0.144400626420975
69 0.144406169652939
70 0.144461944699287
71 0.144472315907478
72 0.144473671913147
73 0.144497185945511
74 0.14446847140789
75 0.144455879926682
76 0.144462257623672
77 0.144465088844299
78 0.144458100199699
79 0.144457831978798
};
\addplot [line width=1.5pt, lightseagreen]
table {%
0 0.028666665777564
1 0.116333335638046
2 0.209999993443489
3 0.416666656732559
4 0.523333311080933
5 0.628666639328003
6 0.761333346366882
7 0.86433333158493
8 0.906000018119812
9 0.934666633605957
10 0.950999975204468
11 0.957666635513306
12 0.960666656494141
13 0.966333329677582
14 0.967000007629395
15 0.968333303928375
16 0.968333303928375
17 0.972000002861023
18 0.970666646957397
19 0.969333350658417
20 0.97133332490921
21 0.969999969005585
22 0.97133332490921
23 0.970333337783813
24 0.972000002861023
25 0.970666646957397
26 0.971666634082794
27 0.968333303928375
28 0.969333350658417
29 0.968666672706604
30 0.967999994754791
31 0.969333350658417
32 0.969999969005585
33 0.970333337783813
34 0.969666659832001
35 0.969999969005585
36 0.969333350658417
37 0.968999981880188
38 0.970333337783813
39 0.968666672706604
40 0.969333350658417
41 0.968666672706604
42 0.969666659832001
43 0.968666672706604
44 0.968666672706604
45 0.969999969005585
46 0.968666672706604
47 0.968999981880188
48 0.968666672706604
49 0.969333350658417
50 0.969333350658417
51 0.968666672706604
52 0.968999981880188
53 0.968999981880188
54 0.968999981880188
55 0.969333350658417
56 0.969333350658417
57 0.969666659832001
58 0.969333350658417
59 0.969333350658417
60 0.969333350658417
61 0.969666659832001
62 0.969333350658417
63 0.969333350658417
64 0.969333350658417
65 0.969333350658417
66 0.969666659832001
67 0.969666659832001
68 0.969666659832001
69 0.969333350658417
70 0.969666659832001
71 0.969333350658417
72 0.969666659832001
73 0.969666659832001
74 0.969666659832001
75 0.969666659832001
76 0.969666659832001
77 0.969666659832001
78 0.969666659832001
79 0.969666659832001
};

\nextgroupplot[
width=.28\linewidth,
xmin=0, xmax=80, ymin=0, ymax=3,
xlabel = {\sf \textit{epochs}}, xlabel style={at={(.5,-.12)}},
title={\sf \textbf{Layers}: 1, \textbf{Digits}: 4},
]
\addplot [line width=1.5pt, steelblue31119180]
table {%
0 16.1500988006592
1 9.3095645904541
2 3.06095433235168
3 2.60867929458618
4 2.55007100105286
5 2.50773167610168
6 2.52670645713806
7 2.44119143486023
8 2.40223932266235
9 2.35879063606262
10 2.25714492797852
11 1.98748922348022
12 1.27430593967438
13 0.938567161560059
14 0.772136270999908
15 0.638410925865173
16 0.570026755332947
17 0.490775108337402
18 0.459385335445404
19 0.434654980897903
20 0.412004470825195
21 0.400936722755432
22 0.398127645254135
23 0.388456076383591
24 0.388161569833755
25 0.385582119226456
26 0.374141991138458
27 0.384999603033066
28 0.38512659072876
29 0.378220170736313
30 0.386807471513748
31 0.384234815835953
32 0.391106456518173
33 0.385352909564972
34 0.390837848186493
35 0.390760451555252
36 0.390821188688278
37 0.400349467992783
38 0.416233658790588
39 0.420717358589172
40 0.406466692686081
41 0.421025425195694
42 0.426708698272705
43 0.433257818222046
44 0.431459188461304
45 0.43684259057045
46 0.444573581218719
47 0.434202462434769
48 0.442012399435043
49 0.452698796987534
50 0.462604761123657
51 0.459520161151886
52 0.474672704935074
53 0.471127927303314
54 0.475456863641739
55 0.476716458797455
56 0.478782892227173
57 0.484319865703583
58 0.484086513519287
59 0.494138330221176
60 0.495136737823486
61 0.496272444725037
62 0.501498639583588
63 0.502588450908661
64 0.50678962469101
65 0.50735992193222
66 0.509675145149231
67 0.510948896408081
68 0.514151930809021
69 0.513838350772858
70 0.514931380748749
71 0.515583395957947
72 0.516949594020844
73 0.517191410064697
74 0.517659068107605
75 0.517891824245453
76 0.518097937107086
77 0.518251657485962
78 0.518297016620636
79 0.518313884735107
};
\addplot [line width=1.5pt, lightseagreen]
table {%
0 0.0529999993741512
1 0.0975999981164932
2 0.191399991512299
3 0.229999989271164
4 0.242399990558624
5 0.252000004053116
6 0.24719999730587
7 0.269600003957748
8 0.279799997806549
9 0.285600006580353
10 0.308999985456467
11 0.376199990510941
12 0.55919998884201
13 0.674799978733063
14 0.730799973011017
15 0.783399999141693
16 0.812399983406067
17 0.844799995422363
18 0.85099995136261
19 0.864799976348877
20 0.870199978351593
21 0.879399955272675
22 0.875
23 0.875400006771088
24 0.875400006771088
25 0.878799974918365
26 0.883599996566772
27 0.882599949836731
28 0.883199989795685
29 0.885199964046478
30 0.884999990463257
31 0.880799949169159
32 0.883599996566772
33 0.881599962711334
34 0.878799974918365
35 0.882799983024597
36 0.880999982357025
37 0.874799966812134
38 0.879399955272675
39 0.873999953269958
40 0.879199981689453
41 0.879599988460541
42 0.876599967479706
43 0.870199978351593
44 0.875999987125397
45 0.872199952602386
46 0.873199999332428
47 0.875
48 0.872999966144562
49 0.872399985790253
50 0.869799971580505
51 0.871800005435944
52 0.867799997329712
53 0.869599997997284
54 0.871399998664856
55 0.872199952602386
56 0.870599985122681
57 0.869799971580505
58 0.870799958705902
59 0.871399998664856
60 0.86899995803833
61 0.867999970912933
62 0.868399977684021
63 0.86899995803833
64 0.868599951267242
65 0.867199957370758
66 0.867399990558624
67 0.866400003433228
68 0.866400003433228
69 0.86679995059967
70 0.866599977016449
71 0.866199970245361
72 0.864999949932098
73 0.864999949932098
74 0.865799963474274
75 0.86599999666214
76 0.86599999666214
77 0.865799963474274
78 0.86599999666214
79 0.86599999666214
};

\nextgroupplot[
width=.28\linewidth,
xmin=0, xmax=80, ymin=0, ymax=3,
xlabel = {\sf \textit{epochs}}, xlabel style={at={(.5,-.12)}},
title={\sf \textbf{Layers}: 1, \textbf{Digits}: 8},
]
\addplot [line width=1.5pt, steelblue31119180]
table {%
0 14.3578033447266
1 3.8737359046936
2 2.54545974731445
3 2.48529529571533
4 2.47089076042175
5 2.46856355667114
6 2.45537662506104
7 2.44946837425232
8 2.44510650634766
9 2.41553831100464
10 2.43618893623352
11 2.40809869766235
12 2.40338063240051
13 2.37879395484924
14 2.38422560691833
15 2.38212370872498
16 2.37861776351929
17 2.37318658828735
18 2.36681127548218
19 2.35814833641052
20 2.36178827285767
21 2.361496925354
22 2.36390781402588
23 2.38373208045959
24 2.35792994499207
25 2.3693311214447
26 2.35754632949829
27 2.36197781562805
28 2.36152505874634
29 2.35618710517883
30 2.35922265052795
31 2.36109733581543
32 2.36173009872437
33 2.36683344841003
34 2.36295104026794
35 2.3664882183075
36 2.36541700363159
37 2.37293648719788
38 2.36519360542297
39 2.38222622871399
40 2.37964797019958
41 2.36569762229919
42 2.37497329711914
43 2.37994503974915
44 2.37199425697327
45 2.37724566459656
46 2.38150119781494
47 2.38069629669189
48 2.38510227203369
49 2.38388133049011
50 2.38761973381042
51 2.38821506500244
52 2.38862228393555
53 2.39248323440552
54 2.38765239715576
55 2.38975262641907
56 2.39051580429077
57 2.39481663703918
58 2.3921046257019
59 2.39409875869751
60 2.3915376663208
61 2.40076923370361
62 2.40027952194214
63 2.40333604812622
64 2.3976948261261
65 2.39705681800842
66 2.40437078475952
67 2.40656232833862
68 2.40609264373779
69 2.40588068962097
70 2.4073007106781
71 2.40962290763855
72 2.40819454193115
73 2.40783739089966
74 2.40781116485596
75 2.40918469429016
76 2.40912628173828
77 2.40904855728149
78 2.40932416915894
79 2.40935015678406
};
\addplot [line width=1.5pt, lightseagreen]
table {%
0 0.0666666701436043
1 0.102111108601093
2 0.161111116409302
3 0.161888897418976
4 0.165444448590279
5 0.159111112356186
6 0.166999995708466
7 0.160222217440605
8 0.172333329916
9 0.166222229599953
10 0.165111109614372
11 0.162888884544373
12 0.177555561065674
13 0.18522222340107
14 0.182444453239441
15 0.178111106157303
16 0.187666669487953
17 0.183333337306976
18 0.189666673541069
19 0.184777781367302
20 0.179444447159767
21 0.182333335280418
22 0.179333329200745
23 0.181999996304512
24 0.184333339333534
25 0.17622222006321
26 0.183888897299767
27 0.183222219347954
28 0.178222224116325
29 0.180111110210419
30 0.177444443106651
31 0.182777777314186
32 0.18133333325386
33 0.181888893246651
34 0.179000005125999
35 0.179555550217628
36 0.180333331227303
37 0.177111119031906
38 0.180111110210419
39 0.182111114263535
40 0.179222226142883
41 0.181999996304512
42 0.182222217321396
43 0.17733334004879
44 0.182333335280418
45 0.176777780056
46 0.177777782082558
47 0.181444451212883
48 0.178333342075348
49 0.180222228169441
50 0.181888893246651
51 0.179888889193535
52 0.180333331227303
53 0.18577778339386
54 0.182777777314186
55 0.182666674256325
56 0.179555550217628
57 0.179666668176651
58 0.181444451212883
59 0.180444449186325
60 0.182444453239441
61 0.18077777326107
62 0.18077777326107
63 0.179222226142883
64 0.181777775287628
65 0.182666674256325
66 0.180111110210419
67 0.178333342075348
68 0.179000005125999
69 0.179444447159767
70 0.180000007152557
71 0.181555554270744
72 0.180555552244186
73 0.182888895273209
74 0.182555556297302
75 0.182333335280418
76 0.182111114263535
77 0.181888893246651
78 0.182222217321396
79 0.182222217321396
};

\nextgroupplot[
width=.28\linewidth,
xmin=0, xmax=80, ymin=0, ymax=3,
xlabel = {\sf \textit{epochs}}, xlabel style={at={(.5,-.12)}},
title={\sf \textbf{Layers}: 1, \textbf{Digits}: 16},
]
\addplot [line width=1.5pt, steelblue31119180]
table {%
0 13.3896608352661
1 2.4637393951416
2 2.41776585578918
3 2.41459488868713
4 2.41029858589172
5 2.40332818031311
6 2.39010691642761
7 2.38682103157043
8 2.36892580986023
9 2.37366390228271
10 2.35293793678284
11 2.3484034538269
12 2.34839034080505
13 2.35122561454773
14 2.33330297470093
15 2.32946991920471
16 2.32257032394409
17 2.31851387023926
18 2.32499313354492
19 2.32041049003601
20 2.31581830978394
21 2.31489753723145
22 2.31470918655396
23 2.31609439849854
24 2.32146263122559
25 2.31520509719849
26 2.31362748146057
27 2.31495547294617
28 2.31490302085876
29 2.31113910675049
30 2.31065988540649
31 2.30958938598633
32 2.3088047504425
33 2.30311727523804
34 2.29875183105469
35 2.28990006446838
36 2.28776574134827
37 2.29134130477905
38 2.28635406494141
39 2.27704215049744
40 2.2708911895752
41 2.28026819229126
42 2.26915049552917
43 2.26785373687744
44 2.28163981437683
45 2.25965714454651
46 2.25188446044922
47 2.24825119972229
48 2.24312520027161
49 2.26840877532959
50 2.27526140213013
51 2.24999070167542
52 2.24077796936035
53 2.2386109828949
54 2.23606657981873
55 2.24958920478821
56 2.23392534255981
57 2.24961733818054
58 2.22815680503845
59 2.226646900177
60 2.22495222091675
61 2.22468113899231
62 2.22432208061218
63 2.2222740650177
64 2.23156690597534
65 2.22099447250366
66 2.22111630439758
67 2.22076296806335
68 2.2199809551239
69 2.2192177772522
70 2.21828413009644
71 2.21872520446777
72 2.21836304664612
73 2.21803283691406
74 2.2178647518158
75 2.21793818473816
76 2.21792387962341
77 2.2178361415863
78 2.21782398223877
79 2.21782827377319
};
\addplot [line width=1.5pt, lightseagreen]
table {%
0 0.0714705884456635
1 0.11011765152216
2 0.132823526859283
3 0.129000008106232
4 0.130882352590561
5 0.133529409766197
6 0.130941182374954
7 0.128470584750175
8 0.130529418587685
9 0.134705886244774
10 0.134117648005486
11 0.137470588088036
12 0.134470596909523
13 0.144117653369904
14 0.142529413104057
15 0.145588234066963
16 0.144176468253136
17 0.143529415130615
18 0.143294125795364
19 0.144941180944443
20 0.144529417157173
21 0.146176472306252
22 0.145882353186607
23 0.144882351160049
24 0.141647055745125
25 0.147647067904472
26 0.143235296010971
27 0.147882357239723
28 0.145235300064087
29 0.146823525428772
30 0.143470585346222
31 0.147176474332809
32 0.151764705777168
33 0.151941180229187
34 0.155411764979362
35 0.157411769032478
36 0.151647061109543
37 0.157882362604141
38 0.16058823466301
39 0.161058828234673
40 0.164294123649597
41 0.16623529791832
42 0.162647068500519
43 0.165647059679031
44 0.165529415011406
45 0.170470595359802
46 0.167470589280128
47 0.170823529362679
48 0.172470599412918
49 0.171294122934341
50 0.16288235783577
51 0.173117652535439
52 0.176882356405258
53 0.176352947950363
54 0.175823539495468
55 0.172411769628525
56 0.178352952003479
57 0.174470588564873
58 0.17964705824852
59 0.180470585823059
60 0.183294117450714
61 0.180470585823059
62 0.181235298514366
63 0.18464706838131
64 0.178470596671104
65 0.182647064328194
66 0.181823536753654
67 0.181823536753654
68 0.183294117450714
69 0.183882355690002
70 0.183352947235107
71 0.18299999833107
72 0.182882353663445
73 0.182823538780212
74 0.182000011205673
75 0.182941183447838
76 0.182529419660568
77 0.183588236570358
78 0.183352947235107
79 0.183294117450714
};

\nextgroupplot[
width=.28\linewidth,
xmin=0, xmax=80, ymin=0, ymax=3,
xlabel = {\sf \textit{epochs}}, xlabel style={at={(.5,-.12)}},
title={\sf \textbf{Layers}: 2, \textbf{Digits}: 2},
]
\addplot [line width=1.5pt, steelblue31119180]
table {%
0 24.4404411315918
1 13.2875127792358
2 5.20647954940796
3 2.54925847053528
4 1.88095319271088
5 1.28719222545624
6 0.691513001918793
7 0.41288086771965
8 0.284831345081329
9 0.207248702645302
10 0.210182130336761
11 0.180913403630257
12 0.186495959758759
13 0.179402127861977
14 0.160388574004173
15 0.161240622401237
16 0.159042432904243
17 0.160917177796364
18 0.159058392047882
19 0.159014627337456
20 0.158749788999557
21 0.159613475203514
22 0.159566521644592
23 0.160102874040604
24 0.159766063094139
25 0.15918280184269
26 0.159575074911118
27 0.159866273403168
28 0.159521296620369
29 0.160297513008118
30 0.159569889307022
31 0.159797504544258
32 0.159519046545029
33 0.159697815775871
34 0.160033211112022
35 0.159811347723007
36 0.159069553017616
37 0.159974843263626
38 0.160136207938194
39 0.15976682305336
40 0.158891603350639
41 0.159282475709915
42 0.159819528460503
43 0.15903602540493
44 0.15924508869648
45 0.15905798971653
46 0.159362345933914
47 0.159403994679451
48 0.158938214182854
49 0.159000501036644
50 0.158628910779953
51 0.158906131982803
52 0.158887311816216
53 0.158724322915077
54 0.158732697367668
55 0.158863231539726
56 0.158664405345917
57 0.158806934952736
58 0.158759281039238
59 0.158662021160126
60 0.158840134739876
61 0.158669501543045
62 0.158548086881638
63 0.158624678850174
64 0.158582150936127
65 0.158590689301491
66 0.158621340990067
67 0.158656343817711
68 0.158537909388542
69 0.158517643809319
70 0.158494263887405
71 0.158518061041832
72 0.158539310097694
73 0.158509373664856
74 0.158506512641907
75 0.158517405390739
76 0.158509626984596
77 0.158510744571686
78 0.158514410257339
79 0.15851528942585
};
\addplot [line width=1.5pt, lightseagreen]
table {%
0 0.00533333327621222
1 0.110333330929279
2 0.285333335399628
3 0.434666663408279
4 0.541333317756653
5 0.68533331155777
6 0.825999975204468
7 0.898000001907349
8 0.931666672229767
9 0.954333305358887
10 0.956666648387909
11 0.960999965667725
12 0.962666630744934
13 0.960999965667725
14 0.970666646957397
15 0.967333316802979
16 0.968333303928375
17 0.967333316802979
18 0.965666651725769
19 0.966333329677582
20 0.967000007629395
21 0.967666685581207
22 0.967000007629395
23 0.967666685581207
24 0.967333316802979
25 0.967999994754791
26 0.967333316802979
27 0.967999994754791
28 0.968333303928375
29 0.968666672706604
30 0.969333350658417
31 0.969333350658417
32 0.969666659832001
33 0.968999981880188
34 0.969333350658417
35 0.969666659832001
36 0.969333350658417
37 0.969333350658417
38 0.968999981880188
39 0.969666659832001
40 0.970666646957397
41 0.969999969005585
42 0.969333350658417
43 0.969999969005585
44 0.969999969005585
45 0.969999969005585
46 0.969333350658417
47 0.969666659832001
48 0.969666659832001
49 0.969666659832001
50 0.969666659832001
51 0.969666659832001
52 0.969666659832001
53 0.969666659832001
54 0.969666659832001
55 0.969666659832001
56 0.969999969005585
57 0.969999969005585
58 0.969999969005585
59 0.969666659832001
60 0.969666659832001
61 0.969999969005585
62 0.969666659832001
63 0.969666659832001
64 0.969666659832001
65 0.969666659832001
66 0.969666659832001
67 0.969666659832001
68 0.969666659832001
69 0.969666659832001
70 0.969666659832001
71 0.969666659832001
72 0.969666659832001
73 0.969666659832001
74 0.969666659832001
75 0.969999969005585
76 0.969999969005585
77 0.969999969005585
78 0.969999969005585
79 0.969999969005585
};

\nextgroupplot[
width=.28\linewidth,
xmin=0, xmax=80, ymin=0, ymax=3,
xlabel = {\sf \textit{epochs}}, xlabel style={at={(.5,-.12)}},
title={\sf \textbf{Layers}: 2, \textbf{Digits}: 4},
]
\addplot [line width=1.5pt, steelblue31119180]
table {%
0 20.7277278900146
1 7.32621049880981
2 2.63171553611755
3 2.49519014358521
4 2.4309093952179
5 2.35248875617981
6 1.72480773925781
7 0.95022451877594
8 0.344149202108383
9 0.201338514685631
10 0.147926151752472
11 0.103682741522789
12 0.104295276105404
13 0.0701083093881607
14 0.101654335856438
15 0.0894699245691299
16 0.11989563703537
17 0.0570755191147327
18 0.0517718680202961
19 0.058745052665472
20 0.0502893030643463
21 0.0341422893106937
22 0.0329722799360752
23 0.0327068045735359
24 0.0296138059347868
25 0.0305431578308344
26 0.0321193709969521
27 0.0348208211362362
28 0.032015573233366
29 0.0324382856488228
30 0.0339152179658413
31 0.0305945910513401
32 0.0318828374147415
33 0.0309576652944088
34 0.0302033480256796
35 0.0316344499588013
36 0.0306083876639605
37 0.0320116281509399
38 0.0300431940704584
39 0.0309958513826132
40 0.0320626012980938
41 0.0319047309458256
42 0.0313069969415665
43 0.0308321807533503
44 0.0305257979780436
45 0.0306009892374277
46 0.031513188034296
47 0.0306771714240313
48 0.0297200940549374
49 0.0310546532273293
50 0.0312414709478617
51 0.0296723004430532
52 0.0299334581941366
53 0.0310395527631044
54 0.030968239530921
55 0.030356477946043
56 0.0307596474885941
57 0.0305406395345926
58 0.0307600218802691
59 0.0309714376926422
60 0.0300596840679646
61 0.0300806816667318
62 0.0301537588238716
63 0.0304604843258858
64 0.0301075614988804
65 0.0307925455272198
66 0.0305134803056717
67 0.0305725485086441
68 0.0304325688630342
69 0.0303461439907551
70 0.0303725618869066
71 0.0302047785371542
72 0.0304144192487001
73 0.0304116681218147
74 0.030364440754056
75 0.0303728953003883
76 0.0303500723093748
77 0.0304156243801117
78 0.0304211936891079
79 0.0304181911051273
};
\addplot [line width=1.5pt, lightseagreen]
table {%
0 0.00939999986439943
1 0.133000001311302
2 0.224399998784065
3 0.253800004720688
4 0.279799997806549
5 0.285199999809265
6 0.434599995613098
7 0.654799997806549
8 0.907399952411652
9 0.944599986076355
10 0.962599992752075
11 0.971799969673157
12 0.973800003528595
13 0.98199999332428
14 0.977199971675873
15 0.978599965572357
16 0.972000002861023
17 0.983599960803986
18 0.984599947929382
19 0.985399961471558
20 0.98719996213913
21 0.990599989891052
22 0.991400003433228
23 0.991999983787537
24 0.991999983787537
25 0.992199957370758
26 0.992799997329712
27 0.99099999666214
28 0.99179995059967
29 0.991599977016449
30 0.990799963474274
31 0.992199957370758
32 0.991199970245361
33 0.991599977016449
34 0.992199957370758
35 0.991400003433228
36 0.992599964141846
37 0.991199970245361
38 0.991999983787537
39 0.991599977016449
40 0.991599977016449
41 0.99179995059967
42 0.992199957370758
43 0.99179995059967
44 0.991999983787537
45 0.991999983787537
46 0.991599977016449
47 0.99179995059967
48 0.992199957370758
49 0.991400003433228
50 0.991400003433228
51 0.991999983787537
52 0.992399990558624
53 0.991999983787537
54 0.99179995059967
55 0.992199957370758
56 0.992199957370758
57 0.992399990558624
58 0.991999983787537
59 0.992199957370758
60 0.99179995059967
61 0.992399990558624
62 0.992399990558624
63 0.992199957370758
64 0.992599964141846
65 0.992199957370758
66 0.992399990558624
67 0.992399990558624
68 0.992199957370758
69 0.992599964141846
70 0.992399990558624
71 0.992399990558624
72 0.992399990558624
73 0.992399990558624
74 0.992399990558624
75 0.992399990558624
76 0.992399990558624
77 0.992399990558624
78 0.992399990558624
79 0.992399990558624
};

\nextgroupplot[
width=.28\linewidth,
xmin=0, xmax=80, ymin=0, ymax=3,
xlabel = {\sf \textit{epochs}}, xlabel style={at={(.5,-.12)}},
title={\sf \textbf{Layers}: 2, \textbf{Digits}: 8},
]
\addplot [line width=1.5pt, steelblue31119180]
table {%
0 18.3998870849609
1 2.60931420326233
2 2.49763607978821
3 2.49152803421021
4 2.48306965827942
5 2.44128227233887
6 2.39632368087769
7 2.3792839050293
8 2.35581707954407
9 2.33973670005798
10 2.32198023796082
11 2.32078433036804
12 2.30661964416504
13 2.29651379585266
14 2.27455425262451
15 2.23982787132263
16 2.23465204238892
17 2.18926191329956
18 2.18876671791077
19 2.1405873298645
20 2.13428568840027
21 1.9650262594223
22 1.00210201740265
23 0.463701009750366
24 0.280223578214645
25 0.173249423503876
26 0.122856922447681
27 0.102980114519596
28 0.0807556360960007
29 0.05643230676651
30 0.056489672511816
31 0.0499426536262035
32 0.0356358736753464
33 0.0324446111917496
34 0.0247795432806015
35 0.0340891964733601
36 0.0238343887031078
37 0.020976236090064
38 0.0157260000705719
39 0.0134950205683708
40 0.0135631570592523
41 0.0137982573360205
42 0.0124598136171699
43 0.0125885894522071
44 0.0130767859518528
45 0.0126475477591157
46 0.0122051350772381
47 0.0120275197550654
48 0.0120146218687296
49 0.0123120751231909
50 0.0116709256544709
51 0.01187829580158
52 0.0116560738533735
53 0.0116751752793789
54 0.0116507392376661
55 0.0116769317537546
56 0.0113241355866194
57 0.0114554055035114
58 0.0113237025216222
59 0.011280732229352
60 0.0113064330071211
61 0.0113094691187143
62 0.0112220076844096
63 0.0112121449783444
64 0.0110119357705116
65 0.0112874219194055
66 0.0111901424825191
67 0.0113532934337854
68 0.011268138885498
69 0.0110221719369292
70 0.0110853761434555
71 0.0110367760062218
72 0.0110564511269331
73 0.0110263926908374
74 0.0110616190358996
75 0.0110742887482047
76 0.0110799660906196
77 0.0110749527812004
78 0.0110669787973166
79 0.0110671957954764
};
\addplot [line width=1.5pt, lightseagreen]
table {%
0 0.0150000005960464
1 0.146999999880791
2 0.155777782201767
3 0.158555552363396
4 0.171666666865349
5 0.187333330512047
6 0.189333334565163
7 0.192333340644836
8 0.197111114859581
9 0.205333337187767
10 0.205222219228745
11 0.207777783274651
12 0.208111107349396
13 0.212444439530373
14 0.218666672706604
15 0.230444446206093
16 0.229222223162651
17 0.24155555665493
18 0.246111109852791
19 0.263111114501953
20 0.26522222161293
21 0.325666666030884
22 0.606333315372467
23 0.846555590629578
24 0.91955554485321
25 0.954888880252838
26 0.966111123561859
27 0.972666680812836
28 0.974222242832184
29 0.983777761459351
30 0.982333362102509
31 0.985666692256927
32 0.989888906478882
33 0.99099999666214
34 0.992333352565765
35 0.990666687488556
36 0.993555545806885
37 0.994000017642975
38 0.9953333735466
39 0.996222257614136
40 0.996333360671997
41 0.995999991893768
42 0.99655556678772
43 0.996333360671997
44 0.996666669845581
45 0.996444463729858
46 0.997111141681671
47 0.996777772903442
48 0.99700003862381
49 0.996888875961304
50 0.997222244739532
51 0.99700003862381
52 0.997111141681671
53 0.996888875961304
54 0.99700003862381
55 0.996888875961304
56 0.997111141681671
57 0.996777772903442
58 0.99700003862381
59 0.997222244739532
60 0.997111141681671
61 0.996888875961304
62 0.996888875961304
63 0.99700003862381
64 0.997111141681671
65 0.997111141681671
66 0.996888875961304
67 0.996777772903442
68 0.996777772903442
69 0.997111141681671
70 0.996777772903442
71 0.996888875961304
72 0.996777772903442
73 0.996777772903442
74 0.996777772903442
75 0.996777772903442
76 0.996777772903442
77 0.996777772903442
78 0.996777772903442
79 0.996777772903442
};

\nextgroupplot[
width=.28\linewidth,
xmin=0, xmax=80, ymin=0, ymax=3,
xlabel = {\sf \textit{epochs}}, xlabel style={at={(.5,-.12)}},
title={\sf \textbf{Layers}: 2, \textbf{Digits}: 16},
]
\addplot [line width=1.5pt, steelblue31119180]
table {%
0 17.4450664520264
1 2.43379902839661
2 2.42226815223694
3 2.42652559280396
4 2.41084098815918
5 2.38659906387329
6 2.3822193145752
7 2.38285613059998
8 2.34139561653137
9 2.33608102798462
10 2.3318977355957
11 2.31934285163879
12 2.31427335739136
13 2.31383323669434
14 2.31490302085876
15 2.31166291236877
16 2.31401896476746
17 2.3107705116272
18 2.31032681465149
19 2.31159329414368
20 2.31320810317993
21 2.31031775474548
22 2.31980657577515
23 2.31332683563232
24 2.31591415405273
25 2.31120824813843
26 2.31073927879333
27 2.31626129150391
28 2.31145286560059
29 2.31509780883789
30 2.31392526626587
31 2.31734538078308
32 2.3129141330719
33 2.31002449989319
34 2.30510640144348
35 2.29889225959778
36 2.28648900985718
37 2.20669293403625
38 0.656595528125763
39 0.448037981987
40 0.316667169332504
41 0.257342964410782
42 0.223648905754089
43 0.198773056268692
44 0.168283954262733
45 0.151079311966896
46 0.123696014285088
47 0.113736361265182
48 0.0884890630841255
49 0.0774477943778038
50 0.0607372745871544
51 0.0509398840367794
52 0.0397918671369553
53 0.0331673435866833
54 0.0277032721787691
55 0.0248166397213936
56 0.0239448025822639
57 0.0204766672104597
58 0.0220616888254881
59 0.0178079474717379
60 0.0171316154301167
61 0.0167137160897255
62 0.0158618725836277
63 0.0151171861216426
64 0.0152756432071328
65 0.014596619643271
66 0.0139954937621951
67 0.0131435189396143
68 0.013577681966126
69 0.0129969231784344
70 0.0141368350014091
71 0.0123135931789875
72 0.0123492758721113
73 0.012202144600451
74 0.0121393678709865
75 0.0122450059279799
76 0.0120175192132592
77 0.0119089111685753
78 0.0118552595376968
79 0.0118497684597969
};
\addplot [line width=1.5pt, lightseagreen]
table {%
0 0.0154705885797739
1 0.133647054433823
2 0.131176471710205
3 0.133411765098572
4 0.131764709949493
5 0.130411773920059
6 0.138470590114594
7 0.136882349848747
8 0.140823528170586
9 0.139058828353882
10 0.144588232040405
11 0.14635294675827
12 0.146823525428772
13 0.147176474332809
14 0.148235291242599
15 0.14341177046299
16 0.145176470279694
17 0.148529410362244
18 0.146470591425896
19 0.14405882358551
20 0.146941184997559
21 0.144588232040405
22 0.14741176366806
23 0.147352948784828
24 0.147235304117203
25 0.143352940678596
26 0.144000008702278
27 0.141411766409874
28 0.145058825612068
29 0.147235304117203
30 0.148235291242599
31 0.148588240146637
32 0.146411761641502
33 0.148705884814262
34 0.152176469564438
35 0.160647064447403
36 0.168411761522293
37 0.195058822631836
38 0.770411789417267
39 0.840294122695923
40 0.915705919265747
41 0.936647057533264
42 0.946529448032379
43 0.953647077083588
44 0.968294143676758
45 0.972235321998596
46 0.980882346630096
47 0.977705895900726
48 0.98694121837616
49 0.986294150352478
50 0.990352988243103
51 0.990764737129211
52 0.99258828163147
53 0.994235336780548
54 0.995058834552765
55 0.99594122171402
56 0.995411813259125
57 0.99576473236084
58 0.995117664337158
59 0.997058868408203
60 0.99700003862381
61 0.996941208839417
62 0.997058868408203
63 0.997529447078705
64 0.997235298156738
65 0.996411800384521
66 0.997647106647491
67 0.998000025749207
68 0.997647106647491
69 0.997352957725525
70 0.99664705991745
71 0.9980588555336
72 0.998000025749207
73 0.998411774635315
74 0.997823536396027
75 0.997941195964813
76 0.997941195964813
77 0.998000025749207
78 0.998176515102386
79 0.99823534488678
};

\nextgroupplot[
width=.28\linewidth,
xmin=0, xmax=80, ymin=0, ymax=3,
xlabel = {\sf \textit{epochs}}, xlabel style={at={(.5,-.12)}},
title={\sf \textbf{Layers}: 3, \textbf{Digits}: 2},
]
\addplot [line width=1.5pt, steelblue31119180]
table {%
0 21.9451274871826
1 9.31511783599854
2 3.88771367073059
3 2.2780704498291
4 1.42700445652008
5 0.786071956157684
6 0.512416422367096
7 0.408473551273346
8 0.309614151716232
9 0.292819380760193
10 0.314926415681839
11 0.268725126981735
12 0.259733140468597
13 0.261062175035477
14 0.261931180953979
15 0.258588910102844
16 0.255600899457932
17 0.252044916152954
18 0.246617168188095
19 0.24476432800293
20 0.240599170327187
21 0.24097865819931
22 0.24372710287571
23 0.239618495106697
24 0.231282025575638
25 0.230042606592178
26 0.23078815639019
27 0.230367347598076
28 0.229712277650833
29 0.229074329137802
30 0.22867077589035
31 0.228100627660751
32 0.227868303656578
33 0.227332592010498
34 0.226973786950111
35 0.226450607180595
36 0.226079136133194
37 0.22576105594635
38 0.225460276007652
39 0.225431680679321
40 0.225103095173836
41 0.224655717611313
42 0.224466472864151
43 0.224346905946732
44 0.224168375134468
45 0.22380019724369
46 0.223682627081871
47 0.223524913191795
48 0.223357766866684
49 0.223072752356529
50 0.222899034619331
51 0.222819849848747
52 0.222597494721413
53 0.222530394792557
54 0.222375839948654
55 0.22223761677742
56 0.222241163253784
57 0.222183600068092
58 0.222021520137787
59 0.221961989998817
60 0.221884861588478
61 0.221812322735786
62 0.221746444702148
63 0.221690073609352
64 0.221685662865639
65 0.221663609147072
66 0.221619516611099
67 0.221567049622536
68 0.221556156873703
69 0.221540659666061
70 0.221518725156784
71 0.221523389220238
72 0.221500679850578
73 0.221492230892181
74 0.22148634493351
75 0.221475854516029
76 0.221474707126617
77 0.221472471952438
78 0.221468150615692
79 0.221468761563301
};
\addplot [line width=1.5pt, lightseagreen]
table {%
0 0.100666664540768
1 0.230999991297722
2 0.345999985933304
3 0.501999974250793
4 0.677333354949951
5 0.809333324432373
6 0.881666660308838
7 0.909999966621399
8 0.934333324432373
9 0.936666667461395
10 0.941333293914795
11 0.951333343982697
12 0.945666670799255
13 0.951333343982697
14 0.951333343982697
15 0.952333331108093
16 0.956333339214325
17 0.954333305358887
18 0.956666648387909
19 0.955999970436096
20 0.955666661262512
21 0.958999991416931
22 0.957333326339722
23 0.957000017166138
24 0.957666635513306
25 0.957666635513306
26 0.957666635513306
27 0.958333313465118
28 0.958000004291534
29 0.958000004291534
30 0.957666635513306
31 0.957666635513306
32 0.958666682243347
33 0.958999991416931
34 0.958666682243347
35 0.958666682243347
36 0.959333300590515
37 0.959666669368744
38 0.959666669368744
39 0.959666669368744
40 0.959666669368744
41 0.960333347320557
42 0.961333334445953
43 0.961333334445953
44 0.961333334445953
45 0.962000012397766
46 0.962000012397766
47 0.962000012397766
48 0.962000012397766
49 0.962000012397766
50 0.96233332157135
51 0.96233332157135
52 0.962000012397766
53 0.96233332157135
54 0.96233332157135
55 0.962000012397766
56 0.96233332157135
57 0.962000012397766
58 0.96233332157135
59 0.96233332157135
60 0.96233332157135
61 0.96233332157135
62 0.96233332157135
63 0.962999999523163
64 0.962999999523163
65 0.962999999523163
66 0.962999999523163
67 0.962999999523163
68 0.962999999523163
69 0.962999999523163
70 0.962999999523163
71 0.962999999523163
72 0.962999999523163
73 0.962999999523163
74 0.962999999523163
75 0.962999999523163
76 0.962999999523163
77 0.962999999523163
78 0.962999999523163
79 0.962999999523163
};

\nextgroupplot[
width=.28\linewidth,
xmin=0, xmax=80, ymin=0, ymax=3,
xlabel = {\sf \textit{epochs}}, xlabel style={at={(.5,-.12)}},
title={\sf \textbf{Layers}: 3, \textbf{Digits}: 4},
]
\addplot [line width=1.5pt, steelblue31119180]
table {%
0 18.7021484375
1 4.93879699707031
2 2.80301666259766
3 2.57306551933289
4 2.58000540733337
5 2.47043251991272
6 2.19960618019104
7 1.39665460586548
8 0.692364394664764
9 0.354639649391174
10 0.214586108922958
11 0.18720355629921
12 0.14168082177639
13 0.118358924984932
14 0.104627154767513
15 0.089441254734993
16 0.0838769376277924
17 0.0755934193730354
18 0.0724425986409187
19 0.0712986811995506
20 0.0708190500736237
21 0.0703265592455864
22 0.0702862665057182
23 0.0708231851458549
24 0.0690239891409874
25 0.0689227059483528
26 0.0684196650981903
27 0.0684515982866287
28 0.0679212883114815
29 0.0689401924610138
30 0.0679810419678688
31 0.066925048828125
32 0.0679359883069992
33 0.0678670704364777
34 0.0670792832970619
35 0.0667172744870186
36 0.0671562626957893
37 0.0671539157629013
38 0.0669883862137794
39 0.0667737796902657
40 0.0668581053614616
41 0.0667601749300957
42 0.0662097260355949
43 0.0662463381886482
44 0.0662015378475189
45 0.0661590993404388
46 0.0659955441951752
47 0.0670070871710777
48 0.0662030801177025
49 0.0657117813825607
50 0.0658419504761696
51 0.0665634796023369
52 0.0654629319906235
53 0.0657985284924507
54 0.0661094263195992
55 0.0660500302910805
56 0.0660656094551086
57 0.0654966160655022
58 0.0659065917134285
59 0.0657034292817116
60 0.0655789226293564
61 0.0655238777399063
62 0.0656369179487228
63 0.0655932947993279
64 0.0655360743403435
65 0.0654848739504814
66 0.0654295459389687
67 0.0654647499322891
68 0.0655065327882767
69 0.0653529316186905
70 0.0654703825712204
71 0.0654998794198036
72 0.0655485466122627
73 0.0654737949371338
74 0.0654626712203026
75 0.0654761791229248
76 0.0654513984918594
77 0.0654461681842804
78 0.06544478982687
79 0.0654469132423401
};
\addplot [line width=1.5pt, lightseagreen]
table {%
0 0.0781999975442886
1 0.171800002455711
2 0.214999988675117
3 0.231600001454353
4 0.240399986505508
5 0.268199980258942
6 0.348399996757507
7 0.537800014019012
8 0.757600009441376
9 0.888399958610535
10 0.93639999628067
11 0.944599986076355
12 0.958999991416931
13 0.967999994754791
14 0.9685999751091
15 0.97599995136261
16 0.977799952030182
17 0.980199992656708
18 0.980399966239929
19 0.97979998588562
20 0.978999972343445
21 0.980599999427795
22 0.980399966239929
23 0.980199992656708
24 0.979999959468842
25 0.980399966239929
26 0.981199979782104
27 0.980799973011017
28 0.980999946594238
29 0.980799973011017
30 0.981199979782104
31 0.981399953365326
32 0.981199979782104
33 0.981399953365326
34 0.981399953365326
35 0.981599986553192
36 0.981399953365326
37 0.981399953365326
38 0.981599986553192
39 0.981199979782104
40 0.981599986553192
41 0.980999946594238
42 0.981199979782104
43 0.981199979782104
44 0.981399953365326
45 0.980999946594238
46 0.981399953365326
47 0.981399953365326
48 0.981199979782104
49 0.981199979782104
50 0.981199979782104
51 0.981599986553192
52 0.981199979782104
53 0.981199979782104
54 0.981399953365326
55 0.981399953365326
56 0.981399953365326
57 0.981599986553192
58 0.98199999332428
59 0.981599986553192
60 0.981799960136414
61 0.98199999332428
62 0.981799960136414
63 0.982199966907501
64 0.982199966907501
65 0.982199966907501
66 0.98199999332428
67 0.98199999332428
68 0.981799960136414
69 0.98199999332428
70 0.98199999332428
71 0.98199999332428
72 0.982199966907501
73 0.98199999332428
74 0.98199999332428
75 0.98199999332428
76 0.98199999332428
77 0.98199999332428
78 0.98199999332428
79 0.98199999332428
};

\nextgroupplot[
width=.28\linewidth,
xmin=0, xmax=80, ymin=0, ymax=3,
xlabel = {\sf \textit{epochs}}, xlabel style={at={(.5,-.12)}},
title={\sf \textbf{Layers}: 3, \textbf{Digits}: 8},
]
\addplot [line width=1.5pt, steelblue31119180]
table {%
0 17.1575164794922
1 2.72470736503601
2 2.53372883796692
3 2.49677443504333
4 2.45859384536743
5 2.45163202285767
6 2.40923476219177
7 2.37166714668274
8 2.32103681564331
9 2.28927445411682
10 2.26015138626099
11 2.22873306274414
12 2.18047904968262
13 1.56851077079773
14 1.0578830242157
15 0.81371808052063
16 0.535475134849548
17 0.340951383113861
18 0.31968367099762
19 0.25122407078743
20 0.229083985090256
21 0.23233588039875
22 0.192200258374214
23 0.191243350505829
24 0.186734348535538
25 0.165568888187408
26 0.155617997050285
27 0.193343549966812
28 0.129384025931358
29 0.138298496603966
30 0.107502967119217
31 0.100252732634544
32 0.0950774177908897
33 0.0838207453489304
34 0.0791037455201149
35 0.0735154375433922
36 0.0639546811580658
37 0.0740374028682709
38 0.0682991370558739
39 0.0599907785654068
40 0.259045153856277
41 0.0564524456858635
42 0.0429968535900116
43 0.0370598174631596
44 0.0367022268474102
45 0.0363026782870293
46 0.0351130850613117
47 0.0350294448435307
48 0.0348507277667522
49 0.0347371511161327
50 0.0347171388566494
51 0.0342464633285999
52 0.0346428975462914
53 0.0332120507955551
54 0.0334892980754375
55 0.0339239053428173
56 0.0338603891432285
57 0.0337865501642227
58 0.0334179475903511
59 0.0337428748607635
60 0.0332477726042271
61 0.0334648862481117
62 0.0334568731486797
63 0.0331694185733795
64 0.0334144160151482
65 0.0332623831927776
66 0.0333565063774586
67 0.0335960052907467
68 0.0332759767770767
69 0.0333943590521812
70 0.0333952493965626
71 0.0332407206296921
72 0.0333398841321468
73 0.0332529246807098
74 0.0332573875784874
75 0.0333063006401062
76 0.0332862809300423
77 0.0332990735769272
78 0.0332948379218578
79 0.0332918502390385
};
\addplot [line width=1.5pt, lightseagreen]
table {%
0 0.0717777758836746
1 0.145444452762604
2 0.151777774095535
3 0.163333341479301
4 0.169444441795349
5 0.171111106872559
6 0.176555559039116
7 0.190555557608604
8 0.198555558919907
9 0.210444450378418
10 0.214555561542511
11 0.225555554032326
12 0.261000007390976
13 0.462666660547256
14 0.581222236156464
15 0.672666668891907
16 0.808333337306976
17 0.891888916492462
18 0.892555594444275
19 0.924555540084839
20 0.931444466114044
21 0.928777813911438
22 0.945111095905304
23 0.942333340644836
24 0.942222237586975
25 0.952444434165955
26 0.958444476127625
27 0.943000018596649
28 0.969777762889862
29 0.961222231388092
30 0.974444448947906
31 0.977222204208374
32 0.978777766227722
33 0.982222259044647
34 0.982888877391815
35 0.981888890266418
36 0.98544442653656
37 0.981333315372467
38 0.980666697025299
39 0.984111130237579
40 0.913666665554047
41 0.984444439411163
42 0.989777803421021
43 0.990888893604279
44 0.990666687488556
45 0.990333318710327
46 0.990666687488556
47 0.990555584430695
48 0.990444481372833
49 0.990000009536743
50 0.989555537700653
51 0.989777803421021
52 0.989666700363159
53 0.989777803421021
54 0.990333318710327
55 0.989888906478882
56 0.989888906478882
57 0.989888906478882
58 0.990000009536743
59 0.989777803421021
60 0.989555537700653
61 0.989444434642792
62 0.990000009536743
63 0.989555537700653
64 0.988888919353485
65 0.989444434642792
66 0.98933333158493
67 0.989222228527069
68 0.989666700363159
69 0.989444434642792
70 0.989000022411346
71 0.989222228527069
72 0.989222228527069
73 0.989111125469208
74 0.989111125469208
75 0.989222228527069
76 0.989222228527069
77 0.989222228527069
78 0.989222228527069
79 0.989222228527069
};

\nextgroupplot[
width=.28\linewidth,
xmin=0, xmax=80, ymin=0, ymax=3,
xlabel = {\sf \textit{epochs}}, xlabel style={at={(.5,-.12)}},
title={\sf \textbf{Layers}: 3, \textbf{Digits}: 16},
]
\addplot [line width=1.5pt, steelblue31119180]
table {%
0 16.1146240234375
1 2.42870950698853
2 2.41445279121399
3 2.41321396827698
4 2.40890455245972
5 2.38319778442383
6 2.37981224060059
7 2.35431504249573
8 2.35861301422119
9 2.36803770065308
10 2.34108066558838
11 2.33065390586853
12 2.31354689598083
13 2.31189322471619
14 2.29623436927795
15 2.38589763641357
16 1.22080755233765
17 0.484020352363586
18 0.280575633049011
19 0.123181484639645
20 0.0792131274938583
21 0.211295172572136
22 0.0912135168910027
23 0.127100273966789
24 0.0757386386394501
25 0.0696318000555038
26 0.0582263730466366
27 0.0401488617062569
28 0.0478412508964539
29 0.0396230444312096
30 0.0344089865684509
31 0.0351892299950123
32 0.0432091616094112
33 0.056397944688797
34 0.0335836410522461
35 0.0301726423203945
36 0.0303497947752476
37 0.0313986018300056
38 0.0280254129320383
39 0.0408926494419575
40 0.0258808545768261
41 0.0378134958446026
42 0.0239765178412199
43 0.0269678179174662
44 0.0284877810627222
45 0.0282676629722118
46 0.031203942373395
47 0.0289275087416172
48 0.0272745136171579
49 0.0232224278151989
50 0.0207971483469009
51 0.0226696785539389
52 0.0235582664608955
53 0.0195689294487238
54 0.0201273988932371
55 0.019157474860549
56 0.0309515632688999
57 0.0195670276880264
58 0.01536850258708
59 0.0157543178647757
60 0.0144152641296387
61 0.0141276102513075
62 0.0142407212406397
63 0.0135445008054376
64 0.0138450330123305
65 0.0138436304405332
66 0.0134198479354382
67 0.0131434313952923
68 0.0132839130237699
69 0.0133889075368643
70 0.0131500018760562
71 0.0132707683369517
72 0.0132649121806026
73 0.0132330060005188
74 0.0133833279833198
75 0.0132223926484585
76 0.0132238892838359
77 0.0132681177929044
78 0.0131941726431251
79 0.0131963090971112
};
\addplot [line width=1.5pt, lightseagreen]
table {%
0 0.0587647072970867
1 0.129176467657089
2 0.131647065281868
3 0.131999999284744
4 0.129764705896378
5 0.131999999284744
6 0.137588232755661
7 0.13335295021534
8 0.13211764395237
9 0.134882360696793
10 0.140117645263672
11 0.142058819532394
12 0.149176478385925
13 0.149294123053551
14 0.153588235378265
15 0.13294118642807
16 0.509588241577148
17 0.796647071838379
18 0.912647068500519
19 0.973647058010101
20 0.982235312461853
21 0.955764710903168
22 0.985823571681976
23 0.966352939605713
24 0.98111766576767
25 0.982705891132355
26 0.983705878257751
27 0.988588273525238
28 0.990705907344818
29 0.990411758422852
30 0.991470634937286
31 0.990058839321136
32 0.987294137477875
33 0.983882367610931
34 0.990235328674316
35 0.990882396697998
36 0.99099999666214
37 0.990235328674316
38 0.991294145584106
39 0.987764716148376
40 0.99241179227829
41 0.988705933094025
42 0.992823541164398
43 0.991882383823395
44 0.990882396697998
45 0.990470588207245
46 0.990764737129211
47 0.991235315799713
48 0.991647064685822
49 0.993235290050507
50 0.993764698505402
51 0.992647051811218
52 0.992705881595612
53 0.993176519870758
54 0.993411779403687
55 0.994000017642975
56 0.991705894470215
57 0.992823541164398
58 0.994823575019836
59 0.994176506996155
60 0.99452942609787
61 0.994823575019836
62 0.994823575019836
63 0.995235323905945
64 0.995117664337158
65 0.994882345199585
66 0.995117664337158
67 0.995117664337158
68 0.995058834552765
69 0.995470583438873
70 0.99558824300766
71 0.995352983474731
72 0.995235323905945
73 0.995176494121552
74 0.995058834552765
75 0.995235323905945
76 0.995235323905945
77 0.995117664337158
78 0.995529413223267
79 0.995470583438873
};
\end{groupplot}

\end{tikzpicture}

    \vspace{-2mm}
    \caption{Test loss and accuracy of $\sf Hyena$ on addition with different numbers of digits and model depths. Each plot reports the results of a different experiment, with the curve tracing test results during training.}
    \label{fig:arithmetic}
\end{figure}